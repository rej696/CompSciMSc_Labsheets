\documentclass[11pt]{article}
\usepackage[utf8]{inputenc}
\usepackage{amsfonts}
\usepackage{amsmath}
\usepackage{float}
\usepackage{tikz}
\usetikzlibrary{automata, positioning, arrows}
\tikzset{
  ->,
  >=stealth',
  node distance=3cm,
  every state/.style={thick, fill=gray!10},
  initial text=$ $,
}

\setlength{\parindent}{0em}
\setlength{\parskip}{1em}

\usepackage{geometry}
\geometry{
  a4paper,
  total={170mm,257mm},
  left=20mm,
  top=20mm,
}

\title{Problem Sheet 6}
\author{Rowan Saunders}
\begin{document}

\begin{titlepage}
	\maketitle
\end{titlepage}

\section{Operations Over Context-Free Languages}

\subsection{Proof that Context-Free Languages are closed over the Union
Operation}
If two languages ($L_1, L_2$) are context-free, there exist corresponding
context-free grammars:
\begin{align*}
	G_1 &= (V_1,\Sigma_1, R_1,S_1)\\
	G_2 &= (V_2,\Sigma_2, R_2,S_2)
\end{align*}

Now, we construct a grammar $G=(V,\Sigma,R,S)$ that generates $L_1 \cup L_2$
where:
\begin{align*}
	V &= V_1 \cup V_2 \cup \{S\} \\
	\Sigma &= \Sigma_1 \cup \Sigma_2 \\
	R &= R_1 \cup R_2 \cup \{S \to S_1, S \to S_2 \}
\end{align*}
Where $S$ is a new start variable

Clearly $G$ generates $L_1 \cup L_2$.
\begin{align*}
	S & \implies S_1 \implies ... (\text{any string from }L(G_1)) \\
	S & \implies S_2 \implies ... (\text{any string from }L(G_2))
\end{align*}
This completes the proof for closure of Context-Free Languages under the union
operation.

\subsection{Proof that Context-Free Languages are closed over the Concatenation
Operation}
If two languages ($L_1, L_2$) are context-free, there exist corresponding
context-free grammars:
\begin{align*}
	G_1 &= (V_1,\Sigma_1, R_1,S_1)\\
	G_2 &= (V_2,\Sigma_2, R_2,S_2)
\end{align*}

Now, we construct a grammar $G=(V,\Sigma,R,S)$ that generates $L_1 \circ L_2$
where:
\begin{align*}
	V &= V_1 \cup V_2 \cup \{S\} \\
	\Sigma &= \Sigma_1 \cup \Sigma_2 \\
	R &= R_1 \cup R_2 \cup \{S \to S_1S_2 \}
\end{align*}
Where $S$ is a new start variable

Clearly $G$ generates $L_1 \circ L_2$.
$$S \implies S_1S_2 \implies (\text{any string from }L(G_1)) \circ (\text{any
string from }L(G_2))$$
This completes the proof for closure of Context-Free Languages under the
concatenation operation.

\newpage
\subsection{Proof that Context-Free Languages are closed over the Kleene Star
Operation}
If a language $L$ is context-free, there exists a corresponding	
context-free grammars:
$$G_1 = (V_1,\Sigma_1, R_1,S_1)$$

Now, we construct a grammar $G=(V,\Sigma,R,S)$ that generates $L^\ast$
where:
\begin{align*}
	V &= V_1 \cup \{S\} \\
	\Sigma &= \Sigma_1 \\
	R &= R_1 \cup \{S \to S_1S, S \to \varepsilon \}
\end{align*}
Where $S$ is a new start variable

Clearly $G$ generates $L^\ast$.
\begin{align*}
	S & \implies \varepsilon \\
	S & \implies S_1S \implies (\text{any string from }L(G_1))	\circ (\text{any
string from }L(G))
\end{align*}
This completes the proof for closure of Context-Free Languages under the
Kleene Star operation.

\end{document}
