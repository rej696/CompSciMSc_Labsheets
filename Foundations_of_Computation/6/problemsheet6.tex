\documentclass[11pt]{article}
\usepackage[utf8]{inputenc}
\usepackage{amsfonts}
\usepackage{amsmath}
\usepackage{float}

\setlength{\parindent}{0em}
\setlength{\parskip}{1em}

\usepackage{geometry}
\geometry{
  a4paper,
  total={170mm,257mm},
  left=20mm,
  top=20mm,
}

\makeatletter
\newcommand{\mathleft}{\@fleqntrue\@mathmargin5pt}
\newcommand{\mathcenter}{\@fleqnfalse}
\makeatother

\title{Problem Sheet 6}
\author{Rowan Saunders}
\begin{document}

\begin{titlepage}
	\maketitle
\end{titlepage}

\section{Operations Over Context-Free Languages}

\subsection{Proof that Context-Free Languages are closed over the Union
Operation}
If two languages ($L_1, L_2$) are context-free, there exist corresponding
context-free grammars:
\begin{align*}
G_1 &= (V_1,\Sigma_1, R_1,S_1)\\
G_2 &= (V_2,\Sigma_2, R_2,S_2)
\end{align*}

Now, we construct a grammar $G=(V,\Sigma,R,S)$ that generates $L_1 \cup L_2$
where:
\begin{align*}
V &= V_1 \cup V_2 \cup \{S\} \\
\Sigma &= \Sigma_1 \cup \Sigma_2 \\
R &= R_1 \cup R_2 \cup \{S \to S_1, S \to S_2 \}
\end{align*}
Where $S$ is a new start variable

Clearly $G$ generates $L_1 \cup L_2$.
\begin{align*}
S & \implies S_1 \implies ... (\text{any string from }L(G_1)) \\
S & \implies S_2 \implies ... (\text{any string from }L(G_2))
\end{align*}
This completes the proof for closure of Context-Free Languages under the union
operation.

\subsection{Proof that Context-Free Languages are closed over the Concatenation
Operation}
If two languages ($L_1, L_2$) are context-free, there exist corresponding
context-free grammars:
\begin{align*}
G_1 &= (V_1,\Sigma_1, R_1,S_1)\\
G_2 &= (V_2,\Sigma_2, R_2,S_2)
\end{align*}

Now, we construct a grammar $G=(V,\Sigma,R,S)$ that generates $L_1 \circ L_2$
where:
\begin{align*}
V &= V_1 \cup V_2 \cup \{S\} \\
\Sigma &= \Sigma_1 \cup \Sigma_2 \\
R &= R_1 \cup R_2 \cup \{S \to S_1S_2 \}
\end{align*}
Where $S$ is a new start variable

Clearly $G$ generates $L_1 \circ L_2$.
$$S \implies S_1S_2 \implies (\text{any string from }L(G_1)) \circ (\text{any
string from }L(G_2))$$
This completes the proof for closure of Context-Free Languages under the
concatenation operation.

\newpage
\subsection{Proof that Context-Free Languages are closed over the Kleene Star
Operation}
If a language $L$ is context-free, there exists a corresponding	
context-free grammars:
$$G_1 = (V_1,\Sigma_1, R_1,S_1)$$

Now, we construct a grammar $G=(V,\Sigma,R,S)$ that generates $L^\ast$
where:
\begin{align*}
V &= V_1 \cup \{S\} \\
\Sigma &= \Sigma_1 \\
R &= R_1 \cup \{S \to S_1S, S \to \varepsilon \}
\end{align*}
Where $S$ is a new start variable

Clearly $G$ generates $L^\ast$.
\begin{align*}
S & \implies \varepsilon \\
S & \implies S_1S \implies (\text{any string from }L(G_1))	\circ (\text{any
string from }L(G))
\end{align*}
This completes the proof for closure of Context-Free Languages under the
Kleene Star operation.

\newpage
\section{Pumping Lemma for Context Free Languages}
Prove the following language is not context-free:
$$L=\{a^p \in \{a\}^\ast \mid p \text{ is a prime number}\}$$

Assume $L$ is context-free, then the context-free pumping lemma applies, with a
pumping length of $n$.

Then consider $a^p$ where $p \geq n$ is a prime number.

Then by the lemma, $a^p = uvxyz$ with the usual conditions:

Let $s = |uxz|$, and $r=|vy|$. Notice that $p=s+r$

Then $|uv^ixy^iz| = s+ ir$, and by the context-free pumping lemma, $a^{s+ir} \in
L$, i.e. $s + ir$ is prime.

Choose a string $w = a^k$ where $k$ is the next prime after the pumping length
$n$ plus $2$. i.e $|w| = n + 2$.

Therefore, given that $|vxy| \leq n$, $|uz| \geq 2$. Therefore, $|uxz| \geq 2$
which is equivalent to $s \geq 2$.

Choose $i = s$, therefore $k = s + ir \implies k = s + sr \implies k = s(r + 1)$

Given that $s \geq 2$, $k$ is not prime, and therefore $w \notin L$

\newpage
\section{Prove that Context-Free Languages are not closed over the Intersection
Operation}

Given two languages:
\begin{align*}
	L_1 &= \{a^k b^k c^j \mid k, j \geq 0 \} \\
	L_2 &= \{a^j b^k c^k \mid k, j \geq 0 \} 
\end{align*}

We construct grammars that accept the languages to prove they are context free.

We construct a grammar $G_1=(V_1,\Sigma_1,R_1,S_1)$ that generates $L_1$
where:
\mathleft
\begin{align*}
	V_1 &= \{S_1, E_1, F_1\} \\
	\Sigma &= \{a, b, c\} \\
	R &= \{ \\
	  & S_1 \to E_1F_1 \\
	  & E_1 \to aE_1b \mid \varepsilon \\
		& F_1 \to cF_1 \mid \varepsilon \\
	\}
\end{align*}
Where $S_1$ is the start variable

We construct a grammar $G_2=(V_2,\Sigma_2,R_2,S_2)$ that generates $L_2$
where:
\begin{align*}
	V_2 &= \{S_2, E_2, F_2\} \\
	\Sigma &= \{a, b, c\} \\
	R &= \{ \\
	  & S_2 \to F_2E_2 \\
	  & E_2 \to bE_2c \mid \varepsilon \\
		& F_2 \to aF_2 \mid \varepsilon \\
	\}
\end{align*}
Where $S_2$ is the start variable

\mathcenter
The language $L = L_1 \cap L_2$, is:
$$L = \{a^k b^k c^k \mid k \geq 0 \}$$

$L$ can be proven to not be a context free language by the pumping lemma.

Assume $L$ is context free, then the pumping lemma applies, which some pumping
length $n$.

Consider $w = a^n b^n c^n$, the length $|w|=3n$ which is greater than $n$, so we
can apply the pumping lemma.
$$w = uvxyz$$

Since $|vxy| \leq n$, one of the following must be true:
\begin{quote}
$vy$ does not contain any $c$ symbols

or 

$vy$ does not contain any $a$ symbols
\end{quote}

Consider $yv^2xy^2z$. The number of $a,b,c$ symbols cannot be equal.

The lemma says $yv^2xy^2z \in L$. This is a contradiction, so $L$ must not be
context-free.

Therefore the context free languages are not closed under the pumping lemma.

\end{document}
